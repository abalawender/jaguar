Spotkania odbywają się w piątki. Do każdego z zadań zostało przydzielonych od dwóch do siedmiu członków grupy, przy czym każde z zadań ma swojego lidera. W przypadku sytuacji spornych podczas realizacji poszczególnych zadań decydujący głos ma lider danego zadania. Jeśli takie podejście nie przyniesie oczekiwanego rozwiązania sprawa może zostać przedstawiona całej grupie i poddana głosowaniu. W sytuacji, gdy konieczna jest szybka decyzja istnieje możliwość rozstrzygnięcia sporu bezpośrednio przez koordynatora projektu, z pominięciem etapu głosowania. Koordynator projektu bierze udział w wyznaczonych zadaniach na zasadach takich samych jak pozostali członkowie grupy projektowej. Dodatkowo pełni on rolę osoby kontaktowej dla ludzi spoza grupy. Do obowiązków koordynatora należą również składanie raportów i koordynowanie prac całego zespołu oraz końcowa ocena członków grupy (w przypadku konfliktów również podlegająca głosowaniu). Koordynacja prac i monitorowanie ich postępów odbywać się będzie na podstawie repozytorium na GitHubie oraz grup projektu na portalu społecznościowym Facebook i w systemie ePortal, w których to na bieżąco przedstawiane i konsultowane są postępy, pytania i wątpliwości. Postępy są nanoszone przez koordynatora na diagram Gantta (lokalnie, planowane jest przeniesienie tego na platformę on-line). Każdy z członków grupy bierze udział w siedmiu spośrod dwunastu zadań wymienionych w harmonogramie prac, ale ma możliwość udzielania się również przy pozostałych pięciu. W związku z takim podejściem każdy z członków grupy projektowej, który ukończy kurs "Projekt Zespołowy" z oceną pozytywną będzie miał takie same prawa własności intelektualnej do uzyskanych wyników prac.
