Koordynacja prac i monitorowanie ich postępów odbywać się będzie na podstawie repozytorium na GitHubie oraz grup projektu na portalu społecznościowym Facebook i w systemie ePortal. Jesteśmy w trakcie ustalania terminu regularnych spotkań. Chcielibyśmy w jakiś sposób dograć go z terminami dostępności dla nas robota Jaguar. Zasady podejmowania decyzji i rozwiązywania konfliktów będą ustalone na najbliższym spotkaniu, podobnie jak zasady przyznawania praw własności intelektualnej do uzyskanych wyników. Na chwilę obecną rola koordynatora sprowadza się do roli osoby kontaktowej dla osób spoza grupy - zakres kompetencji i podział obowiązków ulegną zmianie wraz z chwilą określenia dostępności robota i jego obecnych możliwości oraz wyznaczeniu dokładnych celów projektu.
