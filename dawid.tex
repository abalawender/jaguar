\subsection{Dawid Perdek}
Oprócz Automatyki i Robotyki na Wydziale Elektroniki studiuję również Informatykę na Wydziale Informatyki i Zarządzania (obecnie jestem na urlopie dziekańskim, by móc skupić się na głównym kierunku studiów). Dobrze czuję się w programowaniu, miałem styczność z językami C, C++, Java, Pascal, Delphi, Scala, OCaml oraz Prolog. Umiem odnaleźć się w obsłudze HTML, LaTeXa, pakietu MS Office, systemów operacyjnych Microsoftu czy UNIX. Z przedmiotów związanych z programowaniem zawsze uzyskiwałem oceny z przedziału 4,5 - 5,5. Gorzej u mnie z elektroniką, ale staram się robić postępy w tym kierunku, pomocna jest praca zawodowa przy automatyce przemysłowej. Przedmioty czysto elektroniczne kończyłem z ocenami nie wyższymi niż 3,5. Bardzo podobał mi się jednak kurs Interfejsy Obiektowe (zwłaszcza laboratorium z którego to uzyskałem 5,5), który zakończyłem z oceną bardzo dobrą - projektowa forma zajęć laboratoryjnych bardzo mi odpowiadała co w połączeniu z ciekawą tematyką pomogło mi "wkręcić się" w elektronikę i sprawiło, że zechciałem rozwijać się w tym kierunku - na pewno nie było to bez znaczenia przy wyborze specjalności.
