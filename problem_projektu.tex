Celem realizacji projektu jest dołożenie własnej cegiełki do rozwoju robotyki. Zagadnieniem, którym będziemy się zajmować jest robot mobilny Jaguar, należący do Politechniki Wrocławskiej. Jest to robot nowo zakupiony przez uczelnię w roku 2014. Efektem czego, nikt we Wrocławiu nie miał okazji ani możliwości jego obsługi. Realizując ten projekt, będziemy pierwsi którzy się tego podejmą i przetrą szlaki kolejnym grupom badawczym. Obsługą i praktycznym wykorzystaniem Jaguara zainteresowana jest również firma Neurosoft. Dzięki temu nasza praca nie musi być czysto akademicka, lecz może również uzupełnić pewną lukę w przemyśle. Jest to ewidentny dowód na to, że produkt, którym chcemy się zająć, jest innowacyjny i stwarza duże pole do popisu. Celem projektu jest zrealizowanie algorytmu sterowania robotem mobilnym typu Jaguar. Początkowo w wersji uproszczonej -> „transport z punktu A do punktu B”. Gdy wspomniana wersja zakończy się sukcesem, przewidujemy wraz z firma Neurosoft wyposażyć Jaguara w laser 3D. W naszych zamysłach jest, aby zrealizować algorytm "samodzielnego, bezpiecznego przejeżdżania przez ulicę”. We wspomnianym problemie, robot powinien zauważyć zbliżający się samochód i właściwie zareagować. Oczywiście pomysł ten trzeba będzie skonfrontować z firmą, która dostarczy drogi sprzęt i być może będzie mieć inne plany z jego wykorzystaniem. Problem nasz ociera się o problem „unikania zderzeń” (opis słowny: Wyobraźmy sobie że jesteśmy na lotnisku Heathrow i patrzymy na ludzi z lotu ptaka. Wydawać by się mogło, że wszyscy poruszają się w losowym kierunku tworząc wielki bałagan. Ciekawe jest, że każdy, często zmieniając kierunek ruchu dociera do celu nie zderzywszy się wcześniej z nikim. Jak to możliwe?).\newline
