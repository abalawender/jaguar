\section{Opis projektu}

\subsection{Problem projektu ( mniej niż 1 strona )}
Celem realizacji projektu jest dołożenie własnej cegiełki do rozwoju robotyki. Zagadnieniem, którym będziemy się zajmować jest robot mobilny Jaguar, należący do Politechniki Wrocławskiej. Jest to robot nowo zakupiony przez uczelnię w roku 2014. Efektem czego, nikt we Wrocławiu nie miał okazji ani możliwości jego obsługi. Realizując ten projekt, będziemy pierwsi którzy się tego podejmą i przetrą szlaki kolejnym grupom badawczym. Obsługą i praktycznym wykorzystaniem Jaguara zainteresowana jest również firma Neurosoft. Dzięki temu nasza praca nie musi być czysto akademicka, lecz może również uzupełnić pewną lukę w przemyśle. Jest to ewidentny dowód na to, że produkt, którym chcemy się zająć, jest innowacyjny i stwarza duże pole do popisu. Celem projektu jest zrealizowanie algorytmu sterowania robotem mobilnym typu Jaguar. Początkowo w wersji uproszczonej -> „transport z punktu A do punktu B”. Gdy wspomniana wersja zakończy się sukcesem, przewidujemy wraz z firma Neurosoft wyposażyć Jaguara w laser 3D. W naszych zamysłach jest, aby zrealizować algorytm "samodzielnego, bezpiecznego przejeżdżania przez ulicę”. We wspomnianym problemie, robot powinien zauważyć zbliżający się samochód i właściwie zareagować. Oczywiście pomysł ten trzeba będzie skonfrontować z firmą, która dostarczy drogi sprzęt i być może będzie mieć inne plany z jego wykorzystaniem. Problem nasz ociera się o problem „unikania zderzeń” (opis słowny: Wyobraźmy sobie że jesteśmy na lotnisku Heathrow i patrzymy na ludzi z lotu ptaka. Wydawać by się mogło, że wszyscy poruszają się w losowym kierunku tworząc wielki bałagan. Ciekawe jest, że każdy, często zmieniając kierunek ruchu dociera do celu nie zderzywszy się wcześniej z nikim. Jak to możliwe?).\newline

\subsection{Plan pracy i rozkład w czasie (mniej niż 1 strona)}
\begin{enumerate}
\item[a.] Zapoznanie się ze środowiskiem ROS, zdecentralizowanym systemem kontroli wersji git oraz systemem składu tekstu LaTeX
\item[b.] Wykonanie obudowy dla komputera pokładowego i zamontowanie go na robocie
\item[c.] Zapoznanie z robotem Jaguar, jego dokumentacją i gotowym oprogramowaniem
\item[d.] Uruchomienie robota mobilnego Jaguar w warunkach laboratoryjnych, zapoznanie ze sposobem sterowania
\item[e.] Zainstalowanie środowiska ROS na komputerze pokładowym
\item[f.] Połączenie komputera pokładowego z robotem
\item[g.] Testy działania robota
\item[h.] Opracowanie algorytmu sterowania
\item[i.] Uruchomienie robota w terenie - wykonanie prostego przejazdu z punktu A do punktu B
\item[i.] Wyścigi ?
\item[j.] Ocena użytkownika
\item[k.] Instalacja i konfiguracja dodatkowego sprzętu (laser 3D)
\item[l.] Algorytm sterowania z użyciem czujnika laserowego
\item[m.] Wykonanie dokumentacji
\item[n.] Zarządzanie ?
\end{enumerate}

Wykres Gantta powstanie, gdy potwierdzimy ostateczną wersję listy zadań. (np. program GanttProject )
Kamienie milowe: (zależą od ostatecznej wersji listy zadań)
		
\subsection{Doręczenie (mniej niż 0.5 strony )}
    tabelka
    Oznaczenie	Tydzień	KM	Forma	Tytuł	Jawność
	
	
Doręczenie:(zależą od ostatecznej wersji listy kamieni milowych)			

\subsection{Budżet (mniej niż 0.5 strony )}
	tabelka
	Nr zadania	Potrzeba	Koszt			
													
			
\subsection{Zarządzanie projektem (przydział zadań) (mniej niż 0.5 strony )}
	tabelka
	Nr zadania	Nazwa zadania		Lider	Pozostali członkowie
	
	
\section{Zespół (mniej niż 0.5 strony )}
	Tu będą te opisy który każdy o sobie naskrobie.
	\section{Daria N}
Tu będzie mój opis i każdy stworzy coś takiego o sobie w analogiczny sposób? Dobrze to rozumiem?

	\subsection{Mateusz Tasz}
Opis będzie później.

	\subsection{Dorian Janiak}
Pisze od kilku lat programy w językach C++/C (4.5 z programowania obiektowego).
Podejmował się pracy z takimi językami jak Python, Matlab czy QML.
Obecnie pracuje jako programista C++. 
Jego głównym zainteresowaniem jest grafika 3D (używał OpenGL i GLSL, zna podstawy RayTracingu).
Ukończył kurs języka niemieckiego na poziomie B2 (ocena 5.0).

	\subsection{Dawid Perdek}
Studiuje również Informatykę na Wydziale Informatyki i Zarządzania. Dobrze czuje się w programowaniu, miał styczność z wieloma językami i środowiskami. Lepsze oceny otrzymywał z przedmiotów związanych z programowaniem niż z elektronicznych, ale spodobała mu się elektronika i stara się rozwijać w tym kierunku.

