\section{Opis projektu}

\subsection{Problem projektu}
	Celem realizacji projektu jest dołożenie własnej cegiełki do rozwoju robotyki. Zagadnieniem, którym będziemy się zajmować jest robot mobilny Jaguar, należący do Politechniki Wrocławskiej. Jest to robot nowo zakupiony przez uczelnię w roku 2014. Efektem czego, nikt we Wrocławiu nie miał okazji ani możliwości jego obsługi. Realizując ten projekt, będziemy pierwsi którzy się tego podejmą i przetrą szlaki kolejnym grupom badawczym. Obsługą i praktycznym wykorzystaniem Jaguara zainteresowana jest również firma Neurosoft. Dzięki temu nasza praca nie musi być czysto akademicka, lecz może również uzupełnić pewną lukę w przemyśle. Jest to ewidentny dowód na to, że produkt, którym chcemy się zająć, jest innowacyjny i stwarza duże pole do popisu. Celem projektu jest zrealizowanie algorytmu sterowania robotem mobilnym typu Jaguar. Początkowo w wersji uproszczonej -> „transport z punktu A do punktu B”. Gdy wspomniana wersja zakończy się sukcesem, przewidujemy wraz z firma Neurosoft wyposażyć Jaguara w laser 3D. W naszych zamysłach jest, aby zrealizować algorytm "samodzielnego, bezpiecznego przejeżdżania przez ulicę”. We wspomnianym problemie, robot powinien zauważyć zbliżający się samochód i właściwie zareagować. Oczywiście pomysł ten trzeba będzie skonfrontować z firmą, która dostarczy drogi sprzęt i być może będzie mieć inne plany z jego wykorzystaniem. Problem nasz ociera się o problem „unikania zderzeń” (opis słowny: Wyobraźmy sobie że jesteśmy na lotnisku Heathrow i patrzymy na ludzi z lotu ptaka. Wydawać by się mogło, że wszyscy poruszają się w losowym kierunku tworząc wielki bałagan. Ciekawe jest, że każdy, często zmieniając kierunek ruchu dociera do celu nie zderzywszy się wcześniej z nikim. Jak to możliwe?).\newline

	
\subsection{Plan pracy i rozkład w czasie}
\begin{enumerate}
\item[a.] Zapoznanie się ze środowiskiem ROS, zdecentralizowanym systemem kontroli wersji git oraz systemem składu tekstu LaTeX (do 20.03)
\item[b.] Zapoznanie z robotem Jaguar, jego dokumentacją i gotowym oprogramowaniem (do 31.03)
\item[c.] Zaprojektowanie i wykonanie komputera pokładowego i zamontowanie go w odpowiedniej obudowie na robocie (do 12.04)
\item[d.] Zainstalowanie środowiska ROS na komputerze pokładowym (do 19.04)
\item[e.] Połączenie komputera pokładowego z robotem (do 24.04)
\item[f.] Uruchomienie robota mobilnego Jaguar w warunkach laboratoryjnych, zapoznanie ze sposobem sterowania (do 30.04)
\item[g.] Testy działania robota (K1 - do 10.05)
\item[h.] Opracowanie algorytmu sterowania (do 17.05)
\item[i.] Uruchomienie robota w terenie - wykonanie prostego przejazdu z punktu A do punktu B (K2 - do 24.05)
\item[j.] Instalacja i konfiguracja dodatkowego sprzętu (laser 3D) (do 31.05)
\item[k.] Algorytm sterowania z użyciem czujnika laserowego (do 7.06)
\item[l.] Wykonanie dokumentacji (K3 - do 11.06)
\end{enumerate}

Wykres Gantta stworzony przy pomocy programu GanttProject zostanie przedstawiony w trakcie piątkowego spotkania.
		
%\subsection{Doręczenie (mniej niż 0.5 strony )}
 %   tabelka
  %  Oznaczenie	Tydzień	KM	Forma	Tytuł	Jawność
	
	
%Doręczenie:(zależą od ostatecznej wersji listy kamieni milowych)			

%\subsection{Budżet (mniej niż 0.5 strony )}
%	tabelka
%	Nr zadania	Potrzeba	Koszt			
													
			
\subsection{Zarządzanie projektem}
	Koordynacja prac i monitorowanie ich postępów odbywać się będzie na podstawie repozytorium na GitHubie oraz grup projektu na portalu społecznościowym Facebook i w systemie ePortal. Jesteśmy w trakcie ustalania terminu regularnych spotkań. Chcielibyśmy w jakiś sposób dograć go z terminami dostępności dla nas robota Jaguar. Zasady podejmowania decyzji i rozwiązywania konfliktów będą ustalone na najbliższym spotkaniu, podobnie jak zasady przyznawania praw własności intelektualnej do uzyskanych wyników. Na chwilę obecną rola koordynatora sprowadza się do roli osoby kontaktowej dla osób spoza grupy - zakres kompetencji i podział obowiązków ulegną zmianie wraz z chwilą określenia dostępności robota i jego obecnych możliwości oraz wyznaczeniu dokładnych celów projektu.

%	tabelka
%	Nr zadania	Nazwa zadania		Lider	Pozostali członkowie
	
	
\section{Zespół}
	\subsection{Daria Nowicka}
Dobrze radzi sobie z teoretycznym opisem kinematyki robota (otrzymała ocenę 5,5 z kursu Robotyka 1), lubi programować w Matlabie, nie sprawia jej również trudności obsługa toolboxa Simulink. Oczekuje, że dzięki realizacji projektu będzie miała możliwość praktycznego wykorzystania zdobytej dotychczas wiedzy.
	\subsection{Mateusz Tasz}
Zadania teoretyczne związane z robotyką nie sprawiają mu trudności (kurs robotyka zaliczył na 5.5). Środowisko Matlab zna w stopniu zadowalającym,a przeprowadzanie symulacji komputerowych jest dla niego przyjemnością. Potrafi programować w C/C++ (kurs zaliczony na 5,0), ale zdecydowanie bardziej podoba mu się możliwość wykorzystania tych umiejętności w programowaniu Arduino. Lubi wykonywać zadania majsterkowania.

	\subsection{Dorian Janiak}
Dorian Janiak studiuje na kierunku Automatyka i Robotyka, specjalność Robotyka. 
Pisze od kilku lat programy w języku C++ (ocena 4.5 z programowania obiektowego), języku ANSII C (ocena 4.5 z podstaw programowania).
Posługiwał się również  takimi językami programistycznymi i skryptowymi jak Python, Matlab czy QML.
Obecnie pracuje jako programista C++. Biegle posługuje się pakietem MS Office (ocena 5.5 z technologii informatycznych). 
Jego głównym zainteresowaniem z dziedziny informatyki jest grafika 3D (zna podstawy posługiwania się biblioteką OpenGL wraz z GLSL).
W tym zakresie pisał od podstaw swój własny RayTracer.
Ukończył kurs języka niemieckiego na poziomie B2.  

	\subsection{Dawid Perdek}
Oprócz Automatyki i Robotyki na Wydziale Elektroniki studiuję również Informatykę na Wydziale Informatyki i Zarządzania. Dobrze czuję się w programowaniu, miałem styczność z językami C, C++, Java, Pascal, Delphi, Scala, OCaml, Prolog oraz z Matlabem. Lepsze oceny otrzymywałem z przedmiotów związanych z programowaniem (od 4,5 w górę, z elektronicznych do 3,5), ale spodobała mi się elektronika i staram się rozwijać w tym kierunku.

	\subsection{Adam Balawender}
Mam doświadczenie z językami C, C++, Python oraz LaTeX. Biegle znam środowisko GNU/Linux. Miałem dobre oceny z kursów związanych z programowaniem, statystyką oraz przetwarzaniem sygnałów i obrazów. Nie lubię pisać dokumentacji. Znam język angielski na poziomie C1 oraz podstawy języka niemieckiego, hiszpańskiego i mandaryńskiego. Pracuję jako programista C embedded.

	\subsection{Kamil Bogus}
Zamiłowanie do macierzy i analitycznej analizy kinematyki robotów, potwierdzone ocenami na poziomie 5.0 z Mechaniki Analitycznej czy Robotyki 1, podobnie jeśli chodzi o umiejętności programistyczne, znajomość C, C++, C#, html oraz obsługa programów Matlab i Simulink. Znikome doświadczenie w tworzeniu i montowaniu układów elektronicznych.

	\section{Bartłomiej K}
\subsection{Bartłomiej Kamecki}
Bartłomiej Kamecki studiuje na kierunku Automatyka i Robotyka specjalność Robotyka. Potrafi programować w języku C,C++,Matlab(Oceny z przedmiotów programistycznych w zakresie 3.5-4.5). Dobra znajomość HTML oraz oprogramowań służących tworzeniu dokumentacji takich jak MS Office oraz Latex(Ocena 5.5 z Technologi Informacyjnych oraz 4.5 SCR-Sieci operacyjne). Zagadnienia związane z robotyką takie jak kinematyka robota itp. nie sprawiają mu problemów(Mechanika Analityczna i Robotyka I zaliczone na 4.0) Dobra znajomość języka angielskiego(Angielski B2.2 zaliczony na 5.0) oraz znajomość języka niemieckiego na poziomie podstawowym. 

