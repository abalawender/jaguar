\section{Opis Projektu}
Jako, że nie potrafię się posługiwać Latex, proszę wybaczcie mi brak jakiegokolwiek formatowania.
Teoretycznie do piątku musimy oddać Arentowi opis projektu, który on oceni na 1/4 oceny końcowej. Jeśli dobrze widzę to jeszcze nic nie mamy ...

1.	Problem projektu ( mniej niż 1 strona )
Celem realizacji projektu jest dołożenie własnej cegiełki do rozwoju robotyki. Zagadnieniem, którym będziemy się zajmować jest robot mobilny Jaguar, należący do Politechniki Wrocławskiej. Jest to robot nowo zakupiony przez uczelnię w roku 2014. Efektem czego, nikt we Wrocławiu nie miał okazji ani możliwości jego obsługi. Realizując ten projekt, będziemy pierwsi którzy się tego podejmą i przetrą szlaki kolejnym grupom badawczym. Obsługą i praktycznym wykorzystaniem Jaguara zainteresowana jest również firma XXX. Dzięki czemu nasza praca nie musi być czysto akademicka, ale również uzupełnić pewną lukę w przemyśle. Jest to ewidentny dowód na to, że produkt którym chcemy się zająć jest innowacyjny i stwarzający duże pole do popisu. Celem projektu jest zrealizowanie algorytmu sterowanie robotem mobilnym typu Jaguar, na początku w wersji uproszczonej -> „transport z punktu A do punktu B”. Gdy wspomniana wersja zakończy się sukcesem, przewidujemy wraz z firma X wyposażyć Jaguara w laser 3D. W naszych zamysłach jest aby zrealizować algorytm „samodzielnego przejeżdżania przez ulicę”. We wspomnianym problemie, robot powinien zauważyć zbliżający się samochód i właściwie zareagować. Oczywiście pomysł ten trzeba będzie skonfrontować z firma X, która dostarczy drogi sprzęt i być może będzie mieć inne plany z jego wykorzystaniem. Problem nasz ociera się o problem „unikania zderzeń” (opis słowny: Wyobraźmy sobie że jesteśmy na lotnisku Heathrow i patrzymy na ludzi z lotu ptaka. Wydawać by się mogło, że wszyscy poruszają się w losowym kierunku tworząc wielki bałagan. Ciekawe jest, że każdy, często zmieniając kierunek ruchu dociera do celu nie zderzywszy się wcześniej z nikim. Jak to możliwe?).\newline

2.	Plan pracy i rozkład w czasie (mniej niż 1 strona )
Zadania:\newline
a.	Uruchomienie robota mobilnego Jagua\\
b.	Wykonanie obudowy dla komputera pokładowego\\
c.	Połączenie komputera pokładowego z robotem\\
d.	Zainstalowanie środowiska ROS na komputerze pokładowym\\
e.	Testy działania robota\\
f.	Algorytm sterowania\\
g.	Wyścigi\\
h.	Ocena użytkownika\\
i.	Instalacja i konfiguracja dodatkowego sprzętu (laser 3D )\\
j.	Algorytm sterowania v2\\
k.	Dokumentacja\\
l.	Zarządzanie\\

Wykres Gantta powstanie gdy potwierdzimy ostateczną wersję listy zadań. (np. program GanttProject )
Kamienie milowe: (zależą od ostatecznej wersji listy zadań)
		
3.	Doręczenie (mniej niż 0.5 strony )
    tabelka
    Oznaczenie	Tydzień	KM	Forma	Tytuł	Jawność
	
	
Doręczenie:(zależą od ostatecznej wersji listy kamieni milowych)			

4.	Budżet (mniej niż 0.5 strony )
	tabelka
	Nr zadania	Potrzeba	Koszt			
													
			
5.	Zarządzanie projektem (przydział zadań) (mniej niż 0.5 strony )
	tabelka
	Nr zadania	Nazwa zadania		Lider	Pozostali członkowie
	
	
6.	Zespół (mniej niż 0.5 strony )
	Tu będą te opisy który każdy o sobie naskrobie.
